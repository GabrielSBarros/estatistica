\documentclass[]{article}
\usepackage{lmodern}
\usepackage{amssymb,amsmath}
\usepackage{ifxetex,ifluatex}
\usepackage{fixltx2e} % provides \textsubscript
\ifnum 0\ifxetex 1\fi\ifluatex 1\fi=0 % if pdftex
  \usepackage[T1]{fontenc}
  \usepackage[utf8]{inputenc}
\else % if luatex or xelatex
  \ifxetex
    \usepackage{mathspec}
  \else
    \usepackage{fontspec}
  \fi
  \defaultfontfeatures{Ligatures=TeX,Scale=MatchLowercase}
\fi
% use upquote if available, for straight quotes in verbatim environments
\IfFileExists{upquote.sty}{\usepackage{upquote}}{}
% use microtype if available
\IfFileExists{microtype.sty}{%
\usepackage{microtype}
\UseMicrotypeSet[protrusion]{basicmath} % disable protrusion for tt fonts
}{}
\usepackage[margin=1in]{geometry}
\usepackage{hyperref}
\hypersetup{unicode=true,
            pdftitle={Trabalho de Estatística},
            pdfborder={0 0 0},
            breaklinks=true}
\urlstyle{same}  % don't use monospace font for urls
\usepackage{color}
\usepackage{fancyvrb}
\newcommand{\VerbBar}{|}
\newcommand{\VERB}{\Verb[commandchars=\\\{\}]}
\DefineVerbatimEnvironment{Highlighting}{Verbatim}{commandchars=\\\{\}}
% Add ',fontsize=\small' for more characters per line
\usepackage{framed}
\definecolor{shadecolor}{RGB}{248,248,248}
\newenvironment{Shaded}{\begin{snugshade}}{\end{snugshade}}
\newcommand{\AlertTok}[1]{\textcolor[rgb]{0.94,0.16,0.16}{#1}}
\newcommand{\AnnotationTok}[1]{\textcolor[rgb]{0.56,0.35,0.01}{\textbf{\textit{#1}}}}
\newcommand{\AttributeTok}[1]{\textcolor[rgb]{0.77,0.63,0.00}{#1}}
\newcommand{\BaseNTok}[1]{\textcolor[rgb]{0.00,0.00,0.81}{#1}}
\newcommand{\BuiltInTok}[1]{#1}
\newcommand{\CharTok}[1]{\textcolor[rgb]{0.31,0.60,0.02}{#1}}
\newcommand{\CommentTok}[1]{\textcolor[rgb]{0.56,0.35,0.01}{\textit{#1}}}
\newcommand{\CommentVarTok}[1]{\textcolor[rgb]{0.56,0.35,0.01}{\textbf{\textit{#1}}}}
\newcommand{\ConstantTok}[1]{\textcolor[rgb]{0.00,0.00,0.00}{#1}}
\newcommand{\ControlFlowTok}[1]{\textcolor[rgb]{0.13,0.29,0.53}{\textbf{#1}}}
\newcommand{\DataTypeTok}[1]{\textcolor[rgb]{0.13,0.29,0.53}{#1}}
\newcommand{\DecValTok}[1]{\textcolor[rgb]{0.00,0.00,0.81}{#1}}
\newcommand{\DocumentationTok}[1]{\textcolor[rgb]{0.56,0.35,0.01}{\textbf{\textit{#1}}}}
\newcommand{\ErrorTok}[1]{\textcolor[rgb]{0.64,0.00,0.00}{\textbf{#1}}}
\newcommand{\ExtensionTok}[1]{#1}
\newcommand{\FloatTok}[1]{\textcolor[rgb]{0.00,0.00,0.81}{#1}}
\newcommand{\FunctionTok}[1]{\textcolor[rgb]{0.00,0.00,0.00}{#1}}
\newcommand{\ImportTok}[1]{#1}
\newcommand{\InformationTok}[1]{\textcolor[rgb]{0.56,0.35,0.01}{\textbf{\textit{#1}}}}
\newcommand{\KeywordTok}[1]{\textcolor[rgb]{0.13,0.29,0.53}{\textbf{#1}}}
\newcommand{\NormalTok}[1]{#1}
\newcommand{\OperatorTok}[1]{\textcolor[rgb]{0.81,0.36,0.00}{\textbf{#1}}}
\newcommand{\OtherTok}[1]{\textcolor[rgb]{0.56,0.35,0.01}{#1}}
\newcommand{\PreprocessorTok}[1]{\textcolor[rgb]{0.56,0.35,0.01}{\textit{#1}}}
\newcommand{\RegionMarkerTok}[1]{#1}
\newcommand{\SpecialCharTok}[1]{\textcolor[rgb]{0.00,0.00,0.00}{#1}}
\newcommand{\SpecialStringTok}[1]{\textcolor[rgb]{0.31,0.60,0.02}{#1}}
\newcommand{\StringTok}[1]{\textcolor[rgb]{0.31,0.60,0.02}{#1}}
\newcommand{\VariableTok}[1]{\textcolor[rgb]{0.00,0.00,0.00}{#1}}
\newcommand{\VerbatimStringTok}[1]{\textcolor[rgb]{0.31,0.60,0.02}{#1}}
\newcommand{\WarningTok}[1]{\textcolor[rgb]{0.56,0.35,0.01}{\textbf{\textit{#1}}}}
\usepackage{graphicx,grffile}
\makeatletter
\def\maxwidth{\ifdim\Gin@nat@width>\linewidth\linewidth\else\Gin@nat@width\fi}
\def\maxheight{\ifdim\Gin@nat@height>\textheight\textheight\else\Gin@nat@height\fi}
\makeatother
% Scale images if necessary, so that they will not overflow the page
% margins by default, and it is still possible to overwrite the defaults
% using explicit options in \includegraphics[width, height, ...]{}
\setkeys{Gin}{width=\maxwidth,height=\maxheight,keepaspectratio}
\IfFileExists{parskip.sty}{%
\usepackage{parskip}
}{% else
\setlength{\parindent}{0pt}
\setlength{\parskip}{6pt plus 2pt minus 1pt}
}
\setlength{\emergencystretch}{3em}  % prevent overfull lines
\providecommand{\tightlist}{%
  \setlength{\itemsep}{0pt}\setlength{\parskip}{0pt}}
\setcounter{secnumdepth}{0}
% Redefines (sub)paragraphs to behave more like sections
\ifx\paragraph\undefined\else
\let\oldparagraph\paragraph
\renewcommand{\paragraph}[1]{\oldparagraph{#1}\mbox{}}
\fi
\ifx\subparagraph\undefined\else
\let\oldsubparagraph\subparagraph
\renewcommand{\subparagraph}[1]{\oldsubparagraph{#1}\mbox{}}
\fi

%%% Use protect on footnotes to avoid problems with footnotes in titles
\let\rmarkdownfootnote\footnote%
\def\footnote{\protect\rmarkdownfootnote}

%%% Change title format to be more compact
\usepackage{titling}

% Create subtitle command for use in maketitle
\providecommand{\subtitle}[1]{
  \posttitle{
    \begin{center}\large#1\end{center}
    }
}

\setlength{\droptitle}{-2em}

  \title{Trabalho de Estatística}
    \pretitle{\vspace{\droptitle}\centering\huge}
  \posttitle{\par}
    \author{}
    \preauthor{}\postauthor{}
    \date{}
    \predate{}\postdate{}
  

\begin{document}
\maketitle

\hypertarget{escolha-da-base-de-dados}{%
\section{Escolha da Base de Dados}\label{escolha-da-base-de-dados}}

\hypertarget{para-a-base-de-dados-foram-escolhidas-informacoes-sobre-videos-em-alta-do-youtube-i.e.videos-de-destaque-na-plataforma-postos-em-uma-aba-diferenciada.-para-a-analise-escolhemos-videos-do-youtube-estadunidense.-a-base-encontra-se-no-formato-csv-e-utilizamos-o-software-estatistico-r-para-efetuar-os-comandos-de-maneira-adequada.}{%
\subsubsection{Para a base de dados, foram escolhidas informações sobre
vídeos ``em alta'' do YouTube (i.e.~vídeos de destaque na plataforma,
postos em uma aba diferenciada). Para a análise, escolhemos vídeos do
YouTube estadunidense. A base encontra-se no formato CSV, e utilizamos o
software estatístico R para efetuar os comandos de maneira
adequada.}\label{para-a-base-de-dados-foram-escolhidas-informacoes-sobre-videos-em-alta-do-youtube-i.e.videos-de-destaque-na-plataforma-postos-em-uma-aba-diferenciada.-para-a-analise-escolhemos-videos-do-youtube-estadunidense.-a-base-encontra-se-no-formato-csv-e-utilizamos-o-software-estatistico-r-para-efetuar-os-comandos-de-maneira-adequada.}}

\hypertarget{a-base-e-composta-de-40949-videos-publicados-desde-2006-ate-2018-onde-todos-em-determinado-momento-foram-incluidos-na-aba-em-alta-e-possui-as-seguintes-variaveis}{%
\subsubsection{A base é composta de 40949 vídeos, publicados desde 2006
até 2018, onde todos em determinado momento foram incluídos na aba
``em-alta'', e possui as seguintes
variáveis:}\label{a-base-e-composta-de-40949-videos-publicados-desde-2006-ate-2018-onde-todos-em-determinado-momento-foram-incluidos-na-aba-em-alta-e-possui-as-seguintes-variaveis}}

\hypertarget{identificador-do-video-data-em-que-entrou-na-aba-em-alta-titulo-do-video-titulo-do-canal-que-o-publicou-identificador-da-categoria-do-video-data-e-hora-de-publicacao-do-video-tags-do-video-visualizacoes-do-video-numero-de-gosteis-numero-de-nao-gosteis-contagem-de-comentarios-link-da-imagem-de-thumbnail-do-video-descricao-do-video-e-valores-chaveados-que-dizem-se-os-comentarios-foram-desabilitados-a-avaliacao-do-video-gosteis-e-nao-gosteis-foram-desabilitados-e-por-fim-se-ocorreu-algum-erro-no-video-ou-se-foi-deletado.}{%
\subsubsection{identificador do vídeo, data em que entrou na aba ``em
alta'', título do vídeo, título do canal que o publicou, identificador
da categoria do vídeo, data e hora de publicação do vídeo, tags do
vídeo, visualizações do vídeo, número de ``gosteis'', número de ``não
gosteis'', contagem de comentários, link da imagem de thumbnail do
vídeo, descrição do vídeo e valores chaveados que dizem se os
comentários foram desabilitados, a avaliação do vídeo (``gosteis'' e
``não gosteis'') foram desabilitados, e por fim se ocorreu algum erro no
vídeo ou se foi
deletado.}\label{identificador-do-video-data-em-que-entrou-na-aba-em-alta-titulo-do-video-titulo-do-canal-que-o-publicou-identificador-da-categoria-do-video-data-e-hora-de-publicacao-do-video-tags-do-video-visualizacoes-do-video-numero-de-gosteis-numero-de-nao-gosteis-contagem-de-comentarios-link-da-imagem-de-thumbnail-do-video-descricao-do-video-e-valores-chaveados-que-dizem-se-os-comentarios-foram-desabilitados-a-avaliacao-do-video-gosteis-e-nao-gosteis-foram-desabilitados-e-por-fim-se-ocorreu-algum-erro-no-video-ou-se-foi-deletado.}}

\hypertarget{na-pesquisa-efetuada-sobre-esta-base-de-dados-buscamos-analisar-a-dependencia-das-visualizacoes-de-um-video-com-a-quantidade-de-gosteis-quantidade-de-nao-gosteis-contagem-de-comentario-se-o-video-permite-comentarios-e-se-permite-avaliacoes.}{%
\subsubsection{Na pesquisa efetuada sobre esta base de dados, buscamos
analisar a dependência das visualizações de um vídeo com a quantidade de
``gosteis'', quantidade de ``não gosteis'', contagem de comentário, se o
vídeo permite comentários e se permite
avaliações.}\label{na-pesquisa-efetuada-sobre-esta-base-de-dados-buscamos-analisar-a-dependencia-das-visualizacoes-de-um-video-com-a-quantidade-de-gosteis-quantidade-de-nao-gosteis-contagem-de-comentario-se-o-video-permite-comentarios-e-se-permite-avaliacoes.}}

\hypertarget{analise-descritiva}{%
\section{Análise Descritiva}\label{analise-descritiva}}

\hypertarget{analise-das-correlacoes}{%
\subsection{Análise das Correlações}\label{analise-das-correlacoes}}

\begin{Shaded}
\begin{Highlighting}[]
  \KeywordTok{cor}\NormalTok{(views, likes)}
\end{Highlighting}
\end{Shaded}

\begin{verbatim}
## [1] 0.8491765
\end{verbatim}

\begin{Shaded}
\begin{Highlighting}[]
  \KeywordTok{cor}\NormalTok{(views, comment_count)}
\end{Highlighting}
\end{Shaded}

\begin{verbatim}
## [1] 0.6176213
\end{verbatim}

\begin{Shaded}
\begin{Highlighting}[]
  \KeywordTok{cor}\NormalTok{(views, dislikes)}
\end{Highlighting}
\end{Shaded}

\begin{verbatim}
## [1] 0.4722132
\end{verbatim}

\hypertarget{pode-se-observar-que-a-correlacao-entre-visualizacoes-e-gosteis-e-forte-visto-que-possui-um-valor-entre-070-e-089-porem-quanto-aos-comentarios-e-aos-nao-gosteis-as-correlacoes-sao-moderadas.-visto-que-nao-se-tratam-de-correlacoes-fracas-vale-a-pena-adicionar-as-variaveis-no-modelo}{%
\subsubsection{Pode-se observar que a correlação entre visualizações e
``gosteis'' é forte, visto que possui um valor entre 0,70 e 0,89; porém,
quanto aos comentários e aos ``não gosteis'' as correlações são
moderadas. Visto que não se tratam de correlações fracas, vale a pena
adicionar as variáveis no
modelo}\label{pode-se-observar-que-a-correlacao-entre-visualizacoes-e-gosteis-e-forte-visto-que-possui-um-valor-entre-070-e-089-porem-quanto-aos-comentarios-e-aos-nao-gosteis-as-correlacoes-sao-moderadas.-visto-que-nao-se-tratam-de-correlacoes-fracas-vale-a-pena-adicionar-as-variaveis-no-modelo}}

\begin{Shaded}
\begin{Highlighting}[]
\NormalTok{  limited_views <-}\StringTok{ }\NormalTok{data[views }\OperatorTok{<}\StringTok{ }\DecValTok{3500000}\NormalTok{, ]}
  \KeywordTok{attach}\NormalTok{(limited_views)}
  \KeywordTok{plot}\NormalTok{(views,  }\DataTypeTok{ylab =} \StringTok{"Número de views"}\NormalTok{,  }\DataTypeTok{xlab =} \StringTok{"Index"}\NormalTok{, }\DataTypeTok{main =} \StringTok{"Dispersão das views"}\NormalTok{)}
\end{Highlighting}
\end{Shaded}

\includegraphics{Apresentação_files/figure-latex/unnamed-chunk-5-1.pdf}

\begin{Shaded}
\begin{Highlighting}[]
  \KeywordTok{plot}\NormalTok{(views[}\DecValTok{0}\OperatorTok{:}\DecValTok{1350}\NormalTok{], }\DataTypeTok{ylab =} \StringTok{"Número de views"}\NormalTok{, }\DataTypeTok{xlab =} \StringTok{"Index"}\NormalTok{, }\DataTypeTok{main =} \StringTok{"Dispersão das views"}\NormalTok{)}
\end{Highlighting}
\end{Shaded}

\includegraphics{Apresentação_files/figure-latex/unnamed-chunk-5-2.pdf}

\hypertarget{visto-que-damos-enfoque-para-as-visualizacoes-dos-videos-nesta-base-de-dados-efetuamos-um-grafico-de-dispersao-nessa-variavel.-inicialmente-o-grafico-mostra-se-bastante-poluido-uma-vez-que-ha-muitas-observacoes-de-valores-muitos-proximos.-para-permitir-a-legibilidade-das-informacoes-buscamos-exibir-uma-amostra-de-apenas-1350-observacoes-329-da-amostra-total-no-grafico-de-dispersao-fazemos-isto-apenas-porque-a-amostra-original-e-desornada-pois-caso-o-contrario-teriamos-dados-bastante-enviesados.-podemos-ver-agora-de-maneira-mais-clara-que-as-visualizacoes-dos-videos-em-alta-concentram-se-abaixo-de-1000000-porem-que-ha-ainda-muitos-videos-com-visualizacoes-bem-maiores-e-um-comportamento-de-que-quao-mais-elevada-a-faixa-de-visualizacoes-mais-raros-os-videos-que-a-alcancam-confirmando-algo-esperado-e-intuitivo.}{%
\subsubsection{Visto que damos enfoque para as visualizações dos vídeos
nesta base de dados, efetuamos um gráfico de dispersão nessa variável.
Inicialmente, o gráfico mostra-se bastante poluído, uma vez que há
muitas observações de valores muitos próximos. Para permitir a
legibilidade das informações, buscamos exibir uma amostra de apenas 1350
observações (3,29\% da amostra total) no gráfico de dispersão; fazemos
isto apenas porque a amostra original é desornada, pois caso o
contrário, teríamos dados bastante enviesados. Podemos ver, agora de
maneira mais clara, que as visualizações dos vídeos em alta
concentram-se abaixo de 1000000, porém que há ainda muitos vídeos com
visualizações bem maiores, e um comportamento de que quão mais elevada a
faixa de visualizações, mais raros os vídeos que a alcançam (confirmando
algo esperado e
intuitivo).}\label{visto-que-damos-enfoque-para-as-visualizacoes-dos-videos-nesta-base-de-dados-efetuamos-um-grafico-de-dispersao-nessa-variavel.-inicialmente-o-grafico-mostra-se-bastante-poluido-uma-vez-que-ha-muitas-observacoes-de-valores-muitos-proximos.-para-permitir-a-legibilidade-das-informacoes-buscamos-exibir-uma-amostra-de-apenas-1350-observacoes-329-da-amostra-total-no-grafico-de-dispersao-fazemos-isto-apenas-porque-a-amostra-original-e-desornada-pois-caso-o-contrario-teriamos-dados-bastante-enviesados.-podemos-ver-agora-de-maneira-mais-clara-que-as-visualizacoes-dos-videos-em-alta-concentram-se-abaixo-de-1000000-porem-que-ha-ainda-muitos-videos-com-visualizacoes-bem-maiores-e-um-comportamento-de-que-quao-mais-elevada-a-faixa-de-visualizacoes-mais-raros-os-videos-que-a-alcancam-confirmando-algo-esperado-e-intuitivo.}}

\begin{Shaded}
\begin{Highlighting}[]
\NormalTok{  views_count <-}\StringTok{ }\KeywordTok{subset}\NormalTok{(data[}\DecValTok{8}\NormalTok{], data[}\DecValTok{8}\NormalTok{] }\OperatorTok{<}\StringTok{ }\DecValTok{4000000}\NormalTok{)}
  \KeywordTok{boxplot}\NormalTok{(views_count, }\DataTypeTok{col =} \StringTok{'lightblue'}\NormalTok{, }\DataTypeTok{main =} \StringTok{"Boxplot das views da aba Em alta"}\NormalTok{)}
\end{Highlighting}
\end{Shaded}

\includegraphics{Apresentação_files/figure-latex/unnamed-chunk-7-1.pdf}

\hypertarget{o-boxplot-acima-evidencia-este-comportamento.-embora-a-mediana-dos-valores-de-visualizacao-seja-681861-ha-muitos-videos-com-visualizacoes-muito-altas-levando-o-grafico-consideravelmente-para-baixo-e-com-muitos-outliners-obs-este-grafico-possui-uma-amostra-de-videos-com-menos-de-4-milhoes-de-visualizacoes-visto-que-a-quantidade-de-outliners-na-amostra-e-tao-grande-que-faz-com-que-o-boxplot-da-amostra-original-seja-ilegivel.}{%
\subsubsection{O boxplot acima evidencia este comportamento. Embora a
mediana dos valores de visualização seja 681861, há muitos vídeos com
visualizações muito altas, levando o gráfico consideravelmente para
baixo, e com muitos outliners (OBS: este gráfico possui uma amostra de
vídeos com menos de 4 milhões de visualizações, visto que a quantidade
de outliners na amostra é tão grande, que faz com que o boxplot da
amostra original seja
ilegível).}\label{o-boxplot-acima-evidencia-este-comportamento.-embora-a-mediana-dos-valores-de-visualizacao-seja-681861-ha-muitos-videos-com-visualizacoes-muito-altas-levando-o-grafico-consideravelmente-para-baixo-e-com-muitos-outliners-obs-este-grafico-possui-uma-amostra-de-videos-com-menos-de-4-milhoes-de-visualizacoes-visto-que-a-quantidade-de-outliners-na-amostra-e-tao-grande-que-faz-com-que-o-boxplot-da-amostra-original-seja-ilegivel.}}

\hypertarget{os-dois-graficos-abaixo-referem-se-a-proporcao-de-videos-com-habilitacao-de-comentarios-e-as-quantidades-de-videos-enquadrados-nas-diferentes-categorias.-pelo-grafico-de-setores-podemos-observar-que-a-grande-maioria-dos-videos-em-alta-tem-comentarios-habilitados-ja-pelo-grafico-grafico-de-barras-podemos-observar-que-a-maior-parte-dos-videos-em-alta-pertencem-a-categoria-entretenimento-seguida-pela-categoria-musica.}{%
\subsubsection{Os dois gráficos abaixo referem-se à proporção de vídeos
com habilitação de comentários, e às quantidades de vídeos enquadrados
nas diferentes categorias. Pelo gráfico de setores, podemos observar que
a grande maioria dos vídeos em alta têm comentários habilitados; já pelo
gráfico gráfico de barras, podemos observar que a maior parte dos vídeos
em alta pertencem à categoria ``entretenimento'', seguida pela categoria
``música''.}\label{os-dois-graficos-abaixo-referem-se-a-proporcao-de-videos-com-habilitacao-de-comentarios-e-as-quantidades-de-videos-enquadrados-nas-diferentes-categorias.-pelo-grafico-de-setores-podemos-observar-que-a-grande-maioria-dos-videos-em-alta-tem-comentarios-habilitados-ja-pelo-grafico-grafico-de-barras-podemos-observar-que-a-maior-parte-dos-videos-em-alta-pertencem-a-categoria-entretenimento-seguida-pela-categoria-musica.}}

\begin{Shaded}
\begin{Highlighting}[]
\NormalTok{  freq <-}\StringTok{ }\KeywordTok{table}\NormalTok{(comments_disabled)}
  \KeywordTok{pie}\NormalTok{(freq, }\DataTypeTok{main =} \StringTok{"Habilitação de comentários"}\NormalTok{, }\DataTypeTok{labels =} \KeywordTok{c}\NormalTok{(}\StringTok{"98.45%"}\NormalTok{, }\StringTok{"1.55%"}\NormalTok{), }\DataTypeTok{col=}\KeywordTok{c}\NormalTok{(}\DecValTok{4}\NormalTok{,}\DecValTok{2}\NormalTok{))}
  \KeywordTok{legend}\NormalTok{(}\StringTok{"topright"}\NormalTok{, }\DataTypeTok{fill =} \KeywordTok{c}\NormalTok{(}\DecValTok{4}\NormalTok{,}\DecValTok{2}\NormalTok{), }\DataTypeTok{legend =} \KeywordTok{c}\NormalTok{(}\StringTok{"Habilitados"}\NormalTok{, }\StringTok{"Desativados"}\NormalTok{))}
\end{Highlighting}
\end{Shaded}

\includegraphics{Apresentação_files/figure-latex/unnamed-chunk-8-1.pdf}

\begin{Shaded}
\begin{Highlighting}[]
\NormalTok{  ajustes <-}\StringTok{ }\KeywordTok{table}\NormalTok{(data}\OperatorTok{$}\NormalTok{category_id)}
  \KeywordTok{barplot}\NormalTok{(ajustes, }\DataTypeTok{col =} \StringTok{"lightblue"}\NormalTok{, }\DataTypeTok{main =} \StringTok{"Número de ocorrências de vídeos por categoria"}\NormalTok{,     }\DataTypeTok{ylab =} \StringTok{"Frequência", xlab = "}\NormalTok{Id da categoria}\StringTok{")}
\StringTok{  legend("}\NormalTok{topright}\StringTok{", c("}\DecValTok{10}\OperatorTok{:}\StringTok{ }\NormalTok{Música",}\StringTok{"22: Vlogs "}\NormalTok{,}\StringTok{"23: Comédia"}\NormalTok{, }\StringTok{"24: Entretenimento"}\NormalTok{, }\StringTok{"26: Moda"}\NormalTok{, }\StringTok{"27: Educação"}\NormalTok{), bty =}\StringTok{ "n"}\ErrorTok{)}
\end{Highlighting}
\end{Shaded}

\includegraphics{Apresentação_files/figure-latex/unnamed-chunk-9-1.pdf}

\hypertarget{modelo-e-teste-para-regressao}{%
\section{Modelo e Teste para
Regressão}\label{modelo-e-teste-para-regressao}}

\hypertarget{agora-analisadas-que-ha-correlacao-entre-gosteis-nao-gosteis-e-contagem-de-comentarios-para-com-as-visualizacoes-buscamos-montar-um-modelo-de-regressao-linear-multipla-que-possui-a-seguinte-formulacao}{%
\subsubsection{Agora, analisadas que há correlação entre ``gosteis'',
``não gosteis'' e contagem de comentários para com as visualizações,
buscamos montar um modelo de regressão linear múltipla que possui a
seguinte
formulação:}\label{agora-analisadas-que-ha-correlacao-entre-gosteis-nao-gosteis-e-contagem-de-comentarios-para-com-as-visualizacoes-buscamos-montar-um-modelo-de-regressao-linear-multipla-que-possui-a-seguinte-formulacao}}

\hypertarget{visualizacoes-b0-b1-gosteis-b2-nao-gosteis-b3-contagem-de-comentarios-b4-comentarios-habilitados-b5-avaliacoes-habilitadas}{%
\subsubsection{Visualizações = B0 + B1 * (``Gosteis'') + B2 (``Não
Gosteis'') + B3 * (Contagem de Comentários) + B4 * (Comentários
Habilitados) + B5 * (Avaliações
Habilitadas)}\label{visualizacoes-b0-b1-gosteis-b2-nao-gosteis-b3-contagem-de-comentarios-b4-comentarios-habilitados-b5-avaliacoes-habilitadas}}

\begin{Shaded}
\begin{Highlighting}[]
\NormalTok{frame <-}\StringTok{ }\KeywordTok{data.frame}\NormalTok{(}\DataTypeTok{views =}\NormalTok{ views, }\DataTypeTok{likes =}\NormalTok{ likes, }\DataTypeTok{dislikes =}\NormalTok{ dislikes, }\DataTypeTok{comment_count =}\NormalTok{ comment_count, }\DataTypeTok{comments_disabled =}\NormalTok{ comments_disabled, }\DataTypeTok{ratings_disabled =}\NormalTok{ ratings_disabled)}
\NormalTok{full.model <-}\StringTok{ }\KeywordTok{lm}\NormalTok{(views}\OperatorTok{~}\NormalTok{. , }\DataTypeTok{data =}\NormalTok{ frame)}
\KeywordTok{anova}\NormalTok{(full.model)}
\end{Highlighting}
\end{Shaded}

\begin{verbatim}
## Analysis of Variance Table
## 
## Response: views
##                      Df     Sum Sq    Mean Sq    F value    Pr(>F)    
## likes                 1 1.6144e+18 1.6144e+18 136949.319 < 2.2e-16 ***
## dislikes              1 2.3929e+16 2.3929e+16   2029.975 < 2.2e-16 ***
## comment_count         1 1.1489e+17 1.1489e+17   9746.734 < 2.2e-16 ***
## comments_disabled     1 1.0807e+15 1.0807e+15     91.679 < 2.2e-16 ***
## ratings_disabled      1 1.8428e+15 1.8428e+15    156.329 < 2.2e-16 ***
## Residuals         40943 4.8264e+17 1.1788e+13                         
## ---
## Signif. codes:  0 '***' 0.001 '**' 0.01 '*' 0.05 '.' 0.1 ' ' 1
\end{verbatim}

\hypertarget{com-o-modelo-configurado-efetuamos-o-teste-anova-para-analisar-se-ha-regressao.-na-tabela-podemos-observar-algo-interessante-o-valor-p-para-cada-variavel-e-22-10-16-este-nao-e-so-um-valor-bastante-pequeno-como-tambem-e-o-menor-valor-acima-de-0-que-o-r-consegue-computar-e-representar-em-formato-decimal-double-ou-dupla-precisao.-isto-nos-sugere-que-o-valor-p-e-tao-pequeno-que-representa-que-as-variaveis-sao-significativas-ate-para-uma-pesquisa-com-alfa00-em-outras-palavras-para-uma-pesquisa-com-nivel-de-significancia-00.}{%
\subsubsection{Com o modelo configurado, efetuamos o teste ANOVA para
analisar se há regressão. Na tabela, podemos observar algo interessante:
o valor-p para cada variável é 2,2 * 10\^{}(-16); este não é só um valor
bastante pequeno, como também é o menor valor acima de 0 que o R
consegue computar e representar em formato decimal (double, ou dupla
precisão). Isto nos sugere que o valor-p é tão pequeno que representa
que as variáveis são significativas até para uma pesquisa com alfa=0,0;
em outras palavras, para uma pesquisa com nível de significância
0,0.}\label{com-o-modelo-configurado-efetuamos-o-teste-anova-para-analisar-se-ha-regressao.-na-tabela-podemos-observar-algo-interessante-o-valor-p-para-cada-variavel-e-22-10-16-este-nao-e-so-um-valor-bastante-pequeno-como-tambem-e-o-menor-valor-acima-de-0-que-o-r-consegue-computar-e-representar-em-formato-decimal-double-ou-dupla-precisao.-isto-nos-sugere-que-o-valor-p-e-tao-pequeno-que-representa-que-as-variaveis-sao-significativas-ate-para-uma-pesquisa-com-alfa00-em-outras-palavras-para-uma-pesquisa-com-nivel-de-significancia-00.}}

\hypertarget{podemos-em-seguida-observar-os-valores-f-de-cada-variavel.-o-valor-enorme-para-a-variavel-gosteis-sugere-que-a-mesma-possui-um-impacto-enorme-no-modelo-enquanto-que-o-fato-de-comentarios-estarem-habilitados-impacta-muito-menos-isto-porque-provavelmente-a-grande-massa-de-videos-em-alta-esta-com-os-comentarios-habilitados-fazendo-com-que-a-existencia-da-variavel-em-si-nao-seja-muito-relevante.}{%
\subsubsection{Podemos em seguida observar os valores F de cada
variável. O valor enorme para a variável ``gosteis'' sugere que a mesma
possui um impacto enorme no modelo, enquanto que o fato de comentários
estarem habilitados impacta muito menos (isto porque, provavelmente, a
grande massa de vídeos em alta está com os comentários habilitados,
fazendo com que a existência da variável em si não seja muito
relevante).}\label{podemos-em-seguida-observar-os-valores-f-de-cada-variavel.-o-valor-enorme-para-a-variavel-gosteis-sugere-que-a-mesma-possui-um-impacto-enorme-no-modelo-enquanto-que-o-fato-de-comentarios-estarem-habilitados-impacta-muito-menos-isto-porque-provavelmente-a-grande-massa-de-videos-em-alta-esta-com-os-comentarios-habilitados-fazendo-com-que-a-existencia-da-variavel-em-si-nao-seja-muito-relevante.}}

\hypertarget{em-luz-destas-informacoes-podemos-concluir-com-propriedade-que-ha-regressao-e-que-as-variaveis-independentes-de-fato-influenciam-a-variavel-visualizacoes-e-que-de-fato-compoem-uma-funcao-linear.-portanto-damos-seguimento-ao-experimento.}{%
\subsubsection{Em luz destas informações, podemos concluir com
propriedade que há regressão, e que as variáveis independentes de fato
influenciam a variável Visualizações e que de fato compõem uma função
linear. Portanto, damos seguimento ao
experimento.}\label{em-luz-destas-informacoes-podemos-concluir-com-propriedade-que-ha-regressao-e-que-as-variaveis-independentes-de-fato-influenciam-a-variavel-visualizacoes-e-que-de-fato-compoem-uma-funcao-linear.-portanto-damos-seguimento-ao-experimento.}}

\begin{Shaded}
\begin{Highlighting}[]
 \KeywordTok{library}\NormalTok{(MASS)}
\NormalTok{ step.model <-}\StringTok{ }\KeywordTok{stepAIC}\NormalTok{(full.model, }\DataTypeTok{direction =} \StringTok{"both"}\NormalTok{, }\DataTypeTok{trace =} \OtherTok{FALSE}\NormalTok{)}
 \KeywordTok{summary}\NormalTok{(step.model)}
\end{Highlighting}
\end{Shaded}

\begin{verbatim}
## 
## Call:
## lm(formula = views ~ likes + dislikes + comment_count + comments_disabled + 
##     ratings_disabled, data = frame)
## 
## Residuals:
##       Min        1Q    Median        3Q       Max 
## -39598012   -397415   -168022    180541  86649462 
## 
## Coefficients:
##                         Estimate Std. Error t value Pr(>|t|)    
## (Intercept)            2.092e+05  1.802e+04  11.611  < 2e-16 ***
## likes                  3.558e+01  1.292e-01 275.327  < 2e-16 ***
## dislikes               8.310e+01  8.505e-01  97.704  < 2e-16 ***
## comment_count         -9.770e+01  9.901e-01 -98.670  < 2e-16 ***
## comments_disabledTrue  7.385e+05  1.452e+05   5.086 3.67e-07 ***
## ratings_disabledTrue   3.492e+06  2.793e+05  12.503  < 2e-16 ***
## ---
## Signif. codes:  0 '***' 0.001 '**' 0.01 '*' 0.05 '.' 0.1 ' ' 1
## 
## Residual standard error: 3433000 on 40943 degrees of freedom
## Multiple R-squared:  0.7844, Adjusted R-squared:  0.7844 
## F-statistic: 2.979e+04 on 5 and 40943 DF,  p-value: < 2.2e-16
\end{verbatim}

\hypertarget{em-seguida-efetuamos-o-metodo-stepwise-de-escolha-para-o-melhor-modelo.-estranhamente-o-metodo-stepwise-nao-sugeriu-mudanca-alguma-no-modelo-previo-apenas-replicando-o.}{%
\subsubsection{Em seguida, efetuamos o método stepwise de escolha para o
melhor modelo. Estranhamente, o método stepwise não sugeriu mudança
alguma no modelo prévio, apenas
replicando-o.}\label{em-seguida-efetuamos-o-metodo-stepwise-de-escolha-para-o-melhor-modelo.-estranhamente-o-metodo-stepwise-nao-sugeriu-mudanca-alguma-no-modelo-previo-apenas-replicando-o.}}

\hypertarget{teste-de-hipotese-marginal}{%
\section{Teste de Hipótese Marginal}\label{teste-de-hipotese-marginal}}

\begin{Shaded}
\begin{Highlighting}[]
\NormalTok{  modelo1 =}\StringTok{ }\KeywordTok{lm}\NormalTok{(views }\OperatorTok{~}\StringTok{ }\NormalTok{likes }\OperatorTok{+}\StringTok{ }\NormalTok{dislikes }\OperatorTok{+}\StringTok{ }\NormalTok{comment_count }\OperatorTok{+}\StringTok{ }\NormalTok{comments_disabled }\OperatorTok{+}\StringTok{ }\NormalTok{ratings_disabled)}
\NormalTok{  modelo2 =}\StringTok{ }\KeywordTok{lm}\NormalTok{(views }\OperatorTok{~}\StringTok{ }\NormalTok{likes }\OperatorTok{+}\StringTok{ }\NormalTok{dislikes }\OperatorTok{+}\StringTok{ }\NormalTok{comment_count)}
\end{Highlighting}
\end{Shaded}

\hypertarget{modelo-1}{%
\subsection{Modelo 1}\label{modelo-1}}

\begin{Shaded}
\begin{Highlighting}[]
  \KeywordTok{summary}\NormalTok{(modelo1)}
\end{Highlighting}
\end{Shaded}

\begin{verbatim}
## 
## Call:
## lm(formula = views ~ likes + dislikes + comment_count + comments_disabled + 
##     ratings_disabled)
## 
## Residuals:
##       Min        1Q    Median        3Q       Max 
## -39598012   -397415   -168022    180541  86649462 
## 
## Coefficients:
##                         Estimate Std. Error t value Pr(>|t|)    
## (Intercept)            2.092e+05  1.802e+04  11.611  < 2e-16 ***
## likes                  3.558e+01  1.292e-01 275.327  < 2e-16 ***
## dislikes               8.310e+01  8.505e-01  97.704  < 2e-16 ***
## comment_count         -9.770e+01  9.901e-01 -98.670  < 2e-16 ***
## comments_disabledTrue  7.385e+05  1.452e+05   5.086 3.67e-07 ***
## ratings_disabledTrue   3.492e+06  2.793e+05  12.503  < 2e-16 ***
## ---
## Signif. codes:  0 '***' 0.001 '**' 0.01 '*' 0.05 '.' 0.1 ' ' 1
## 
## Residual standard error: 3433000 on 40943 degrees of freedom
## Multiple R-squared:  0.7844, Adjusted R-squared:  0.7844 
## F-statistic: 2.979e+04 on 5 and 40943 DF,  p-value: < 2.2e-16
\end{verbatim}

\hypertarget{modelo-2}{%
\subsection{Modelo 2}\label{modelo-2}}

\begin{Shaded}
\begin{Highlighting}[]
  \KeywordTok{summary}\NormalTok{(modelo2)}
\end{Highlighting}
\end{Shaded}

\begin{verbatim}
## 
## Call:
## lm(formula = views ~ likes + dislikes + comment_count)
## 
## Residuals:
##       Min        1Q    Median        3Q       Max 
## -39604136   -400942   -191417    165894  86661356 
## 
## Coefficients:
##                 Estimate Std. Error t value Pr(>|t|)    
## (Intercept)    2.374e+05  1.792e+04   13.25   <2e-16 ***
## likes          3.555e+01  1.296e-01  274.30   <2e-16 ***
## dislikes       8.316e+01  8.529e-01   97.50   <2e-16 ***
## comment_count -9.773e+01  9.929e-01  -98.43   <2e-16 ***
## ---
## Signif. codes:  0 '***' 0.001 '**' 0.01 '*' 0.05 '.' 0.1 ' ' 1
## 
## Residual standard error: 3444000 on 40945 degrees of freedom
## Multiple R-squared:  0.7831, Adjusted R-squared:  0.7831 
## F-statistic: 4.928e+04 on 3 and 40945 DF,  p-value: < 2.2e-16
\end{verbatim}

\hypertarget{visto-que-na-tabela-anova-do-modelo-anterior-nos-presenciamos-que-os-valores-f-mais-baixos-pertenciam-as-variaveis-comentarios-habilitados-e-avaliacoes-habilitadas-criamos-um-modelo-alternativo-sem-estas-duas-variaveis-e-buscamos-compara-los-a-seguir.}{%
\subsubsection{Visto que na tabela ANOVA do modelo anterior nós
presenciamos que os valores F mais baixos pertenciam às variáveis
Comentários Habilitados e Avaliações Habilitadas, criamos um modelo
alternativo sem estas duas variáveis, e buscamos compará-los a
seguir.}\label{visto-que-na-tabela-anova-do-modelo-anterior-nos-presenciamos-que-os-valores-f-mais-baixos-pertenciam-as-variaveis-comentarios-habilitados-e-avaliacoes-habilitadas-criamos-um-modelo-alternativo-sem-estas-duas-variaveis-e-buscamos-compara-los-a-seguir.}}

\hypertarget{efetuando-o-teste-de-hipotese-marginal-para-cada-parametro-beta-em-ambos-os-modelos-podemos-observar-que-no-primeiro-modelo-todas-as-variaveis---com-excecao-de-comentarios-habilitados---possuem-o-menor-valor-p-representavel-tornando-as-todas-relevantes.-quanto-ao-segundo-modelo-de-maneira-quase-identica-todas-as-variaveis-possuem-o-menor-valor-p-representavel.-desta-maneira-nos-resta-tomar-como-criterio-de-decisao-utilizar-o-r-ou-coeficiente-de-determinacao.-podemos-observar-que-ambos-os-modelos-conseguem-explicar-a-maioria-das-observacoes-de-visualizacoes-porem-que-o-primeiro-modelo-tem-maior-coeficiente-assim-escolhemos-o-primeiro-modelo-como-o-melhor-modelo-ate-entao.}{%
\subsubsection{Efetuando o teste de hipótese marginal para cada
parâmetro beta em ambos os modelos, podemos observar que no primeiro
modelo, todas as variáveis - com exceção de Comentários Habilitados -
possuem o menor valor-p representável, tornando-as todas relevantes.
Quanto ao segundo modelo, de maneira quase idêntica, todas as variáveis
possuem o menor valor-p representável. Desta maneira, nos resta tomar
como critério de decisão utilizar o R² (ou coeficiente de determinação).
Podemos observar que ambos os modelos conseguem explicar a maioria das
observações de visualizações, porém que o primeiro modelo tem maior
coeficiente; assim, escolhemos o primeiro modelo como o melhor modelo
até
então.}\label{efetuando-o-teste-de-hipotese-marginal-para-cada-parametro-beta-em-ambos-os-modelos-podemos-observar-que-no-primeiro-modelo-todas-as-variaveis---com-excecao-de-comentarios-habilitados---possuem-o-menor-valor-p-representavel-tornando-as-todas-relevantes.-quanto-ao-segundo-modelo-de-maneira-quase-identica-todas-as-variaveis-possuem-o-menor-valor-p-representavel.-desta-maneira-nos-resta-tomar-como-criterio-de-decisao-utilizar-o-r-ou-coeficiente-de-determinacao.-podemos-observar-que-ambos-os-modelos-conseguem-explicar-a-maioria-das-observacoes-de-visualizacoes-porem-que-o-primeiro-modelo-tem-maior-coeficiente-assim-escolhemos-o-primeiro-modelo-como-o-melhor-modelo-ate-entao.}}

\hypertarget{e-interessante-notar-que-embora-tenhamos-julgado-que-as-variaveis-comentarios-habilitados-e-avaliacoes-habilitadas-pouco-importassem-para-o-modelo-elas-na-realidade-colaboram-com-a-melhoria-do-componente-deterministico-do-modelo-tornando-o-melhor.}{%
\subsubsection{É interessante notar que, embora tenhamos julgado que as
variáveis Comentários Habilitados e Avaliações Habilitadas pouco
importassem para o modelo, elas na realidade colaboram com a melhoria do
componente determinístico do modelo, tornando-o
melhor.}\label{e-interessante-notar-que-embora-tenhamos-julgado-que-as-variaveis-comentarios-habilitados-e-avaliacoes-habilitadas-pouco-importassem-para-o-modelo-elas-na-realidade-colaboram-com-a-melhoria-do-componente-deterministico-do-modelo-tornando-o-melhor.}}

\hypertarget{analise-de-residuos}{%
\section{Análise de Resíduos}\label{analise-de-residuos}}

\hypertarget{modelo-1-1}{%
\subsection{Modelo 1}\label{modelo-1-1}}

\begin{Shaded}
\begin{Highlighting}[]
  \KeywordTok{shapiro.test}\NormalTok{(}\KeywordTok{residuals}\NormalTok{(modelo1)[}\DecValTok{0}\OperatorTok{:}\DecValTok{4500}\NormalTok{])}
\end{Highlighting}
\end{Shaded}

\begin{verbatim}
## 
##  Shapiro-Wilk normality test
## 
## data:  residuals(modelo1)[0:4500]
## W = 0.44594, p-value < 2.2e-16
\end{verbatim}

\hypertarget{seguimos-entao-para-a-validacao-das-pressuposicoes-de-um-modelo-de-regressao-linear-que-consiste-em-testar-se-os-residuos-comportam-se-de-maneira-esperada.}{%
\subsubsection{Seguimos então para a validação das pressuposições de um
modelo de regressão linear, que consiste em testar se os resíduos
comportam-se de maneira
esperada.}\label{seguimos-entao-para-a-validacao-das-pressuposicoes-de-um-modelo-de-regressao-linear-que-consiste-em-testar-se-os-residuos-comportam-se-de-maneira-esperada.}}

\hypertarget{iniciamos-com-o-teste-shapiro-wilk-de-normalidade-dos-residuos-i.e.se-os-residuos-do-modelo-seguem-uma-distribuicao-normal.-limitamos-o-tamanho-da-amostra-para-4500-observacoes-dada-uma-limitacao-do-teste-em-r.-para-nossa-decepcao-tivemos-o-infortunio-de-que-o-teste-shapiro-wilk-entregasse-nos-um-valor-p-baixissimo-o-menor-valor-p-representavel-o-que-nos-causou-estranheza.-visto-que-o-valor-p-e-baixissimo-nao-e-possivel-aceitar-a-hipotese-h0-de-que-os-residuos-seguem-uma-distribuicao-normal-fazendo-com-que-as-pressuposicoes-de-um-modelo-de-regressao-linear-sejam-quebradas-consequentemente-o-modelo-nao-e-confiavel.}{%
\subsubsection{Iniciamos com o teste Shapiro-Wilk de normalidade dos
resíduos (i.e.~se os resíduos do modelo seguem uma distribuição normal).
Limitamos o tamanho da amostra para 4500 observações, dada uma limitação
do teste em R. Para nossa decepção, tivemos o infortúnio de que o teste
Shapiro-Wilk entregasse-nos um valor-p baixíssimo: o menor valor-p
representável, o que nos causou estranheza. Visto que o valor-p é
baixíssimo, não é possível aceitar a hipótese H0 de que os resíduos
seguem uma distribuição normal, fazendo com que as pressuposições de um
modelo de regressão linear sejam quebradas; consequentemente, o modelo
não é
confiável.}\label{iniciamos-com-o-teste-shapiro-wilk-de-normalidade-dos-residuos-i.e.se-os-residuos-do-modelo-seguem-uma-distribuicao-normal.-limitamos-o-tamanho-da-amostra-para-4500-observacoes-dada-uma-limitacao-do-teste-em-r.-para-nossa-decepcao-tivemos-o-infortunio-de-que-o-teste-shapiro-wilk-entregasse-nos-um-valor-p-baixissimo-o-menor-valor-p-representavel-o-que-nos-causou-estranheza.-visto-que-o-valor-p-e-baixissimo-nao-e-possivel-aceitar-a-hipotese-h0-de-que-os-residuos-seguem-uma-distribuicao-normal-fazendo-com-que-as-pressuposicoes-de-um-modelo-de-regressao-linear-sejam-quebradas-consequentemente-o-modelo-nao-e-confiavel.}}

\begin{Shaded}
\begin{Highlighting}[]
  \KeywordTok{qqnorm}\NormalTok{(}\KeywordTok{residuals}\NormalTok{(modelo1), }\DataTypeTok{ylab=}\StringTok{"Resíduos"}\NormalTok{, }\DataTypeTok{xlab =} \StringTok{"Quantis Teóricos")}
\StringTok{  qqline(residuals(modelo1))}
\end{Highlighting}
\end{Shaded}

\includegraphics{Apresentação_files/figure-latex/unnamed-chunk-16-1.pdf}

\hypertarget{para-checar-o-resultado-do-valor-p-retornado-pelo-teste-shapiro-wilk-efetuamos-um-grafico-de-residuos-vs-quantis-teoricos-onde-podemos-observar-sinuosidade-no-grafico-mostrando-uma-grande-variabilidade-nos-residuos-que-destoa-bastante-do-comportamento-teorico-expresso-na-linha.}{%
\subsubsection{Para checar o resultado do valor-p retornado pelo teste
Shapiro-Wilk, efetuamos um gráfico de Resíduos vs Quantis Teóricos, onde
podemos observar sinuosidade no gráfico, mostrando uma grande
variabilidade nos resíduos, que destoa bastante do comportamento teórico
expresso na
linha.}\label{para-checar-o-resultado-do-valor-p-retornado-pelo-teste-shapiro-wilk-efetuamos-um-grafico-de-residuos-vs-quantis-teoricos-onde-podemos-observar-sinuosidade-no-grafico-mostrando-uma-grande-variabilidade-nos-residuos-que-destoa-bastante-do-comportamento-teorico-expresso-na-linha.}}

\hypertarget{modelo-2-1}{%
\subsection{Modelo 2}\label{modelo-2-1}}

\hypertarget{visto-que-o-primeiro-modelo-foi-entao-rejeitado-pelo-teste-das-pressuposicoes-na-analise-residual-nos-restou-analisar-o-segundo-modelo-visto-anteriormente.-efetuando-o-teste-shapiro-wilk-tambem-neste-segundo-modelo-podemos-ver-um-comportamento-muito-similar-nos-residuos-o-que-invalida-tambem-este-segundo-modelo-tratando-se-de-uma-verdadeira-desventura.}{%
\subsubsection{Visto que o primeiro modelo foi então rejeitado pelo
teste das pressuposições na análise residual, nos restou analisar o
segundo modelo visto anteriormente. Efetuando o teste Shapiro-Wilk
também neste segundo modelo, podemos ver um comportamento muito similar
nos resíduos, o que invalida também este segundo modelo, tratando-se de
uma verdadeira
desventura.}\label{visto-que-o-primeiro-modelo-foi-entao-rejeitado-pelo-teste-das-pressuposicoes-na-analise-residual-nos-restou-analisar-o-segundo-modelo-visto-anteriormente.-efetuando-o-teste-shapiro-wilk-tambem-neste-segundo-modelo-podemos-ver-um-comportamento-muito-similar-nos-residuos-o-que-invalida-tambem-este-segundo-modelo-tratando-se-de-uma-verdadeira-desventura.}}

\begin{Shaded}
\begin{Highlighting}[]
  \KeywordTok{shapiro.test}\NormalTok{(}\KeywordTok{residuals}\NormalTok{(modelo2)[}\DecValTok{0}\OperatorTok{:}\DecValTok{4500}\NormalTok{])}
\end{Highlighting}
\end{Shaded}

\begin{verbatim}
## 
##  Shapiro-Wilk normality test
## 
## data:  residuals(modelo2)[0:4500]
## W = 0.42851, p-value < 2.2e-16
\end{verbatim}

\begin{Shaded}
\begin{Highlighting}[]
  \KeywordTok{qqnorm}\NormalTok{(}\KeywordTok{residuals}\NormalTok{(modelo2), }\DataTypeTok{ylab=}\StringTok{"Resíduos"}\NormalTok{, }\DataTypeTok{xlab =} \StringTok{"Quantis Teóricos")}
\StringTok{  qqline(residuals(modelo2))}
\end{Highlighting}
\end{Shaded}

\includegraphics{Apresentação_files/figure-latex/unnamed-chunk-18-1.pdf}

\hypertarget{algoritmo-de-selecao-dos-videos-em-alta}{%
\section{Algoritmo de Seleção dos Vídeos em
Alta}\label{algoritmo-de-selecao-dos-videos-em-alta}}

\hypertarget{para-um-video-entrar-na-aba-em-alta-o-algoritmo-leva-em-consideracao-muitos-sinais-incluindo-mas-nao-limitados-a}{%
\subsubsection{Para um vídeo entrar na aba em alta, o algoritmo leva em
consideração muitos sinais, incluindo (mas não limitados
a):}\label{para-um-video-entrar-na-aba-em-alta-o-algoritmo-leva-em-consideracao-muitos-sinais-incluindo-mas-nao-limitados-a}}

\hypertarget{contagem-de-visualizacoes}{%
\subsubsection{- Contagem de
visualizações}\label{contagem-de-visualizacoes}}

\hypertarget{a-taxa-de-crescimento-em-visualizacoes}{%
\subsubsection{- A taxa de crescimento em
visualizações}\label{a-taxa-de-crescimento-em-visualizacoes}}

\hypertarget{de-onde-vem-as-visualizacoes-incluindo-fora-do-youtube}{%
\subsubsection{- De onde vêm as visualizações (incluindo fora do
YouTube)}\label{de-onde-vem-as-visualizacoes-incluindo-fora-do-youtube}}

\hypertarget{a-data-de-publicacao-do-video-quanto-mais-recente-maior-e-a-chance-de-entrar-na-aba-em-alta}{%
\subsubsection{- A data de publicação do vídeo (quanto mais recente,
maior é a chance de entrar na aba em
alta)}\label{a-data-de-publicacao-do-video-quanto-mais-recente-maior-e-a-chance-de-entrar-na-aba-em-alta}}

\hypertarget{esse-algoritmo-e-automatico-no-entanto-tambem-e-verificado-se-os-videos-sugeridos-pelo-algoritmo-sao}{%
\subsubsection{Esse algoritmo é automático, no entanto, também é
verificado se os vídeos sugeridos pelo algoritmo
são:}\label{esse-algoritmo-e-automatico-no-entanto-tambem-e-verificado-se-os-videos-sugeridos-pelo-algoritmo-sao}}

\hypertarget{atraentes-para-uma-ampla-gama-de-espectadores}{%
\subsubsection{- Atraentes para uma ampla gama de
espectadores}\label{atraentes-para-uma-ampla-gama-de-espectadores}}

\hypertarget{nao-enganosas-clickbaits-ou-sensacionalistas}{%
\subsubsection{- Não enganosas, clickbaits ou
sensacionalistas}\label{nao-enganosas-clickbaits-ou-sensacionalistas}}

\hypertarget{surpreendentes-novos-e-originais}{%
\subsubsection{- Surpreendentes, novos e
originais}\label{surpreendentes-novos-e-originais}}

\hypertarget{essa-etapa-e-feita-manualmente-pelos-funcionarios-do-youtube.}{%
\subsubsection{Essa etapa é feita manualmente pelos funcionários do
YouTube.}\label{essa-etapa-e-feita-manualmente-pelos-funcionarios-do-youtube.}}

\hypertarget{conclusao}{%
\section{Conclusão}\label{conclusao}}

\hypertarget{apos-todos-os-testes-foi-observado-que-os-modelos-propostos-de-regressao-linear-multipla-nao-sao-validos-visto-que-as-pressuposicoes-para-que-os-mesmos-ocorram-nao-sao-satisfeitas.-nos-estranhou-inicialmente-o-comportamento-que-os-residuos-apresentam.-entretanto-ao-analisar-mais-profundamente-o-procedimento-efetuado-pelo-youtube-para-que-um-video-da-plataforma-entre-na-aba-em-alta-podemos-ver-que-nao-apenas-ha-o-componente-automatizado-como-tambem-ha-uma-forte-influencia-manual-humana-de-uma-equipe-de-curadoria-do-youtube-como-melhor-descrito-na-secao-acima.-desta-maneira-acreditamos-que-parte-da-componente-residual-do-modelo-de-regressao-linear-se-da-gracas-a-interferencia-humana-nas-selecoes-dos-videos.}{%
\subsubsection{Após todos os testes foi observado que os modelos
propostos de regressão linear múltipla não são válidos, visto que as
pressuposições para que os mesmos ocorram não são satisfeitas. Nos
estranhou, inicialmente, o comportamento que os resíduos apresentam.
Entretanto, ao analisar mais profundamente o procedimento efetuado pelo
YouTube para que um vídeo da plataforma entre na aba ``em alta'',
podemos ver que não apenas há o componente automatizado, como também há
uma forte influência manual humana de uma equipe de curadoria do
YouTube, como melhor descrito na seção acima. Desta maneira, acreditamos
que parte da componente residual do modelo de regressão linear se dá
graças à interferência humana nas seleções dos
vídeos.}\label{apos-todos-os-testes-foi-observado-que-os-modelos-propostos-de-regressao-linear-multipla-nao-sao-validos-visto-que-as-pressuposicoes-para-que-os-mesmos-ocorram-nao-sao-satisfeitas.-nos-estranhou-inicialmente-o-comportamento-que-os-residuos-apresentam.-entretanto-ao-analisar-mais-profundamente-o-procedimento-efetuado-pelo-youtube-para-que-um-video-da-plataforma-entre-na-aba-em-alta-podemos-ver-que-nao-apenas-ha-o-componente-automatizado-como-tambem-ha-uma-forte-influencia-manual-humana-de-uma-equipe-de-curadoria-do-youtube-como-melhor-descrito-na-secao-acima.-desta-maneira-acreditamos-que-parte-da-componente-residual-do-modelo-de-regressao-linear-se-da-gracas-a-interferencia-humana-nas-selecoes-dos-videos.}}


\end{document}
